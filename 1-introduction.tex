% !TEX root = ./main.tex
\chapter{Introduction}

This chapter discusses fundamental properties of aerosols and computational modeling techniques which motivate this thesis. A description of atmospheric aerosols and the challenges associated with capturing the complexity of aerosol properties and their environmental feedbacks is discussed. Additionally, numerical modeling treatments for aerosols are presented along with approaches to improve the characterization of aerosol complexity. Finally, research questions are presented which outline primary avenues of inquiry for this thesis.  

\section{The complexity of aerosols and environmental feedbacks}\label{aerosol_properties}

An aerosol is a collection of particles composed of one or more chemical species that are suspended in a fluid or gas. In the atmosphere, aerosol particles vary considerably in terms of their physical properties such as size, composition, and origin. Additionally, the chemical, thermodynamic, and radiative properties of aerosol particles can alter the state of the aerosol and the surrounding environment through numerous feedback mechanisms. In turn, aerosol particles exhibit a complex, non-linear coupling with the environment that spans broad spatial and temporal scales.

Aerosol particles are typically measured by their diameter where spherical morphology is assumed. The smallest particles have diameters on the order of 1 nm and are produced via the nucleation of low-volatility vapors. On the opposite extreme of particle sizes, the largest particle diameters can exceed 100 $\upmu$m. In total, aerosols span approximately five orders of magnitude. To capture the broad scale of particle diameters that may be present in a population of aerosol particles, aerosol size distributions often represent the number concentration of particles as a function of the logarithm of particle diameter. The particle size distribution may be represented by multiple modes---lognormal size distributions---that are differentiated by the characteristics of particles within each mode, including growth and removal mechanisms. Typically, three distinct modes are present in a particle size distribution: the nucleation, accumulation, and coarse mode. 

Nucleation mode particles are up to 20 nm in diameter and undergo rapid growth as gas-phase species condense onto the particle surface or as particles inelastically collide through coagulation. They are removed from the nucleation mode by growth within \hl{XX} time into the accumulation mode, which spans particle diameters from 0.1 $\upmu$m to 2 $\upmu$m . In addition to particles that enter the accumulation mode through growth by condensation or coagulation, particles may be released directly into the accumulation mode via primary emissions. Removal mechanisms such as wet and dry deposition are least efficient in the accumulation mode, allowing particles to remain suspended in the atmosphere for days to weeks. Particles in the coarse mode have diameters exceeding 2 $\upmu$m  and are produced by mechanical processes such as abrasion and the resuspension of dust. Due to their size, particles in the coarse mode are rapidly removed by gravitational settling within minutes to hours. This multi-modal description of the aerosol size distribution points to the inherent complexity of aerosol population dynamics---production, growth, and removal mechanisms differ considerably by particle size. 

As noted, production mechanisms vary across aerosol modes (e.g., nucleation of low-volatility vapors, emission of primary aerosol  into the accumulation mode, resuspension of coarse particles, etc.). These processes typically involve different chemical species. For example, whereas volatile organic compounds (VOCs) such as isoprene and other organic carbon (OC) species may undergo oxidation reactions which lower their volatility and promote particle nucleation, particles released directly into the accumulation or coarse mode as primary aerosol may consist of either organics that are produced during combustion such as black carbon (BC) or inorganics such as sea salt spray, mineral dust, \hl{[other species]}. \hl{Here its worth acknowledging contribution of precursor emissions to chemical aging, secondary production of aerosol-phase matter, changes to aerosol mixing state, etc.}. As a result, aerosol particles are compositionally diverse. 

In addition to diversity in the composition of aerosol particles across the size distribution, aerosol populations also exhibit spatiotemporal variations which alter the local structure and composition of the aerosol. The geographic distribution of emission sources, varied land use, and topography lead to spatial heterogeneities in the emission of gas-phase precursors and primary aerosols. Additionally, temporal trends alter the meteorological state of the atmosphere and the concentration of reactive gas or aerosol-phase species. For instance, diurnal variation in the structure of the boundary layer due to surface heating determines the strength of vertical transport and mixing of primary aerosol or reactive gas-phase species. \hl{[Could talk about photolysis]}. Furthermore, the timing of emissions may play a crucial role in determining whether a chemical reaction will take place; reactive species must be present in the same space and time to undergo reaction.

\cite{seinfeld_atmospheric_1998}

\section{Impacts of aerosols on climate}

Aerosols alter the Earth's radiative budget directly through scattering and absorption of shortwave (solar) radiation. The scattering of solar radiation by aerosols back out to space increases planetary albedo, thereby decreasing the intensity of radiation reaching the Earth's surface (\cite{charlson_climate_1969}; \cite{charlson_climate_1992}). As a result, scattering generally contributes a net cooling effect. The intensity of scattering depends on the composition of the aerosol, with strongly-scattering species including sulfate and nitrate. Aerosols may also absorb broadband radiation, re-emitting in the form of thermal radiation that results in a net warming effect. Absorption varies by aerosol species; strongly absorbing species include carbonaceous aerosol such as black carbon and elemental carbon. Scattering and absorption of solar radiation due to aerosols alters the stability of the atmosphere due to changes in the vertical profile of temperature (\cite{li_scattering_2022}; \cite{lau_observational_2006}). 

In addition to direct aerosol-radiative effects, aerosols also alter the climate through indirect effects with clouds commonly referred to as aerosol-cloud interactions. Hygroscopic aerosol particles act as cloud condensation nuclei (CCN), thereby allowing water vapor to condense onto their surface at ambient supersaturations $S$ typical of the troposphere ($S\lesssim1\%$). \cite{twomey_influence_1977} was the first to note that higher concentrations of CCN result in a greater abundance of small cloud droplets. This in turn leads to an increase in cloud albedo, causing greater reflection of solar radiation back to space and thus a net cooling effect on climate. In addition to the Twomey effect, the impact of aerosol number concentration on droplet size can delay or prevent the onset of collision-coalescence necessary to initiate precipitation. This effect was first discovered by \cite{albrecht_aerosols_1989} and enhances the lifetime of clouds, thereby prolonging the reflection of solar radiation. 

The global mean effective radiative forcing (ERF) due to the combination of direct and indirect effects is estimated by the Intergovernmental Panel on Climate Change (IPCC) to be in the range of -2.0 to -0.6 \si{W.m^{-2}} within 95\% confidence, with a mean of -1.3 \si{W.m^{-2}} (\cite{ipcc_report_2021}). Separating the ERF into forcing due to aerosol-cloud interactions and direct radiative forcing, aerosol-cloud interactions contribute the largest magnitude of forcing in the range -1.7 to -0.3 \si{W.m^{-2}} with a mean of -1.0 \si{W.m^{-2}}. Direct effects contribute -0.6 to 0.0 \si{W.m^{-2}} with a mean of -0.3 \si{W.m^{-2}}. 

The magnitude of uncertainty in ERF due to aerosol direct and indirect effects remains large due to a host of factors. As discussed in Section \ref{aerosol_properties}, aerosol particle size and composition are highly varied and determine climate-relevant properties including a particle's scattering and absorption coefficients and its hygroscopicity. Representing the full range of aerosol composition and properties in a modeling framework is highly computationally expensive and current state-of-the-science global scale climate models use simplified aerosol treatments such as sectional or modal models (aerosol model treatments are discussed in more detail in Section \ref{aerosol_model_treatments}). Furthermore, estimates for ERF due to aerosol-cloud interactions in particular are poorly constrained due to limited understanding of the coupling between microphysical phenomena and cloud macrophysical structure for deep convective clouds where phase transitions complicate the role of aerosols and thermodynamic feedbacks (\cite{fan_review_2016}). In addition, aerosols are highly spatially heterogeneous due to localized sources, resulting in varied concentrations and properties that determine the local activity of CCN and associated aerosol-cloud interactions.

\section{Spatial heterogeneity of aerosols}

There exists a well established link between surface spatial heterogeneities and their impacts on the evolution of the atmospheric state. For instance, \cite{fast_impact_2019} conducted a joint observation and modeling study to evaluate the role of soil moisture heterogeneity in promoting deeply convecting clouds. The authors compared observations collected during the Holistic Interactions of Shallow Clouds, Aerosols, and Land-Ecosystems (HI-SCALE) campaign against a set of large-eddy simulations (LES) where the spatial heterogeneity of soil moisture was varied from a constant distribution to higher variability which closely matched the observed soil moisture spatial heterogeneity.  The authors found that under modeling scenarios with smoothly varying soil moisture, clouds did not develop into open cell, deep convective cumulus capable of precipitating and instead were characterized by shallow, uniform non-precipitating clouds. In order to replicate the degree of cloud heterogeneity and the development of deeply convecting clouds observed during the HI-SCALE campaign, realistic spatial variability in the modeled soil moisture distribution was required. 

In addition to soil moisture fluxes, spatial heterogeneity in surface heat fluxes has been shown to be critical to the development of atmospheric circulation. \cite{lee_effect_2019} conducted an idealized LES study in which surface heat fluxes (including both sensible and latent heat flux) were prescribed by checkerboard patterns of ranging spatial heterogeneity (most heterogeneous being the lowest frequency checkerboard pattern with the largest pattern length scale, and the least heterogeneous being the highest frequency patterns with the smallest pattern length scale). The authors found that secondary circulation developed under scenarios with the highest spatial heterogeneity and minimal background winds (less than 2 \si{m.s^{-1}}). This circulation was responsible for transporting moisture from checkerboard regions with greater latent heat flux to drier regions with lesser latent heat flux. 

The spatial distribution of primary aerosol emissions and precursor gas phase emissions results in spatially varying concentrations that span orders of magnitude and complex variability in the composition of aerosols. For example, urban aerosol number concentrations are highly variable; whereas a significant number of nucleation mode particles ($\sim10^5$--$10^6$ \si{cm^{-3}}) may be found nearby busy highways, the concentration of nucleation mode particles is significantly reduced downwind of the highway due in large part to coagulation (\cite{zhu_study_2002}). By comparison, rural aerosol concentrations are more spatially uniform and lower in number with concentrations ranging between $\sim10^3$--$10^4$ \si{cm^{-3}}. Whereas urban aerosol are composed of a mixture of primary carbonacous aerosol released from vehicular and industrial combustion and species resulting from gas-particle partitioning of emitted gas phase compounds such as NO$_x$ or SO$_2$, rural aerosol contain a large fraction of organics resulting from the oxidation of biogenic volatile organic compounds (BVOCs) in the gas phase to form secondary organic aerosol (SOA). In rural regions containing significant amounts of agricultural land use, ammonium may also be present in elevated fractions (\cite{seinfeld_atmospheric_1998}). 

The spatial heterogeneity of both gas phase and aerosol number concentrations impacts how particles age due to concentration dependent processes such as coagulation and gas-particle partitioning. For a number distribution $n(v,t)$ that is a function of particle volume $v$ and time $t$, the rate of change to the number distribution due to coagulation is defined as 
\begin{equation}
\frac{\partial n(v, t)}{\partial t} = \frac{1}{2}\int_0^{v}K(v-v', v')n(v-v', t)n(v', t)dv' - n(v,t)\int_0^{\infty}K(v',v)n(v',t)dv',
\label{eq:coag}
\end{equation}
where $K(v_1, v_2)$ is the coagulation kernel between particles of volume $v_1$ and $v_2$. The first term on the right hand side of Equation \ref{eq:coag} is coagulation gain while the second term is coagulation loss. Note how each term is proportional to the square of the number distribution. This causes the rate of coagulation to be highly sensitive to changes in aerosol number concentration, whereby highly polluted regions (such as nearby highway emissions) experience elevated rates of coagulation. 

In addition to coagulation, the rate of chemical reactions in both the gas phase and gas-particle partitioning are concentration dependent and thus the spatial heterogeneity of emitted compounds  determines the effective rate at which such reactions proceed. An extensive body of literature evaluates the effects of chemical segregation (i.e., the degree to which precursor compounds are spatially separated or collocated) on the abundance of reaction products in the atmospheric boundary layer (\cite{schumann_large-eddy_1989}; \cite{sykes_turbulent_1994}; \cite{molemaker_control_1998}; \cite{krol_effects_2000}; \cite{vinuesa_fluxes_2003}; \cite{auger_chemical_2007}; \cite{pugh_influence_2011}; \cite{ouwersloot_segregation_2011}; \cite{dlugi_balances_2014}; \cite{kim_impact_2016}; \cite{li_error_2021}; \cite{wang_segregation_2022}). All of these studies focus on second order gas phase reactions which are prevalent in atmospheric chemistry and utilize LES to resolve turbulence-chemistry interactions. Initial studies focused on generic species and a range of imposed reaction rates (\cite{schumann_large-eddy_1989}; \cite{sykes_turbulent_1994};  \cite{molemaker_control_1998}). Subsequently, modeling studies have investigated the production and destruction of ozone and oxidation of generic VOCs (\cite{krol_effects_2000}; \cite{auger_chemical_2007}) and more recently oxidation of isoprene by OH (\cite{pugh_influence_2011}; \cite{ouwersloot_segregation_2011}; \cite{dlugi_balances_2014}; \cite{kim_impact_2016}). Advances in computing have allowed the use of direct numerical simulations of gas phase reactions in the planetary boundary layer (\cite{li_error_2021}) and the modeling of entire urban regions with LES to evaluate chemical segregation (\cite{wang_segregation_2022}). Note that these studies do not model aerosols, however the coupling between the gas phase and aerosols through gas-particle partitioning suggests  chemical segregation due to the spatial heterogeneity of emissions likely influences the aerosol state. %including composition and climate relevant properties such as optical properties and CCN activity. 


\begin{itemize}

\item Impacts on aerosol processes: non-linear, concentration dependent processes
\begin{itemize}
\item Coagulation
\item Chemistry
\begin{itemize}
\item Present body of literature investigates predominantly gas-phase chemistry (e.g., isoprene-OH reaction) and contribution of chemical segregation due to turbulence and spatial distribution of reactive species
\end{itemize}
\item These processes alter the radiative and hygroscopic properties of the aerosol 
\end{itemize}

\item Representation of aerosol spatial heterogeneity in models
\begin{itemize}
\item Lack of resolution, impact of sub-grid scale variability on aerosol properties (Radiative properties, CCN concentrations)
\begin{itemize}
\item Fast et al. 2022 (properties)
\item Lin et al. 2017 find that aerosol (mass I think) SGV over the pacific ocean within a typical GCM grid cell is 15\% near the surface and as high as 50\% in the free troposphere. (this could also be cited in the paragraph above on variability of aerosols).
\item Weigum et al. 2016 look at the effect of SGV on AOD and CCN by comparing WRF-Chem runs at 80 km resolution against 10 km resolution and find an underestimation of AOD by 13\% and an overestimation of CCN by 27\%. They find gas-phase chemistry and aerosol water uptake are processes which are most affected.
\item Qian et al. 2010 (study over mexico, contribution of emissions to SGV)
\item Gustafason et al. 2011 (similar to Qian 2010, radiative effects)
\item Crippa et al. 2017
\end{itemize}
\item Attempts at parameterizations (adaptive grids, plume in grid modeling, coagulation parameterizations, PDF-based methods, stochastic fields, etc.)
\end{itemize}

\end{itemize}

\section{Treatment of aerosols across modeling frameworks}\label{aerosol_model_treatments}

\begin{itemize}

\item Simplified bulk, modal, sectional treatments
\begin{itemize}
\item Use in regional, global scale models
\item Consequences of simplified treatment on representation of CCN concentrations, etc. 
\end{itemize}
\item Particle resolved aerosol modeling
\item Transport representation 
\begin{itemize}
\item Large scale models use RANS and thus cannot resolve turbulence and associated heterogeneities in gas, aerosol concentrations
\item LES - adoption for modeling aerosols is still nascent, some examples like UCLALES-SALSA, DALES have been used to evaluate aerosol-cloud interactions, but none leverage a high-resolution particle resolved aerosol treatment
\end{itemize}

\end{itemize}

\section{Objectives of this thesis}

\begin{itemize}

\item Primary science questions
\begin{itemize}
\item Impacts of emissions spatial heterogeneity on aerosol processes (e.g., coagulation, chemistry, etc.)?
\item Impacts of emissions spatial heterogeneity on aerosol properties (e.g., composition, concentration, hygroscopicity)? 
\item How is the impact of spatial heterogeneity on aerosols modulated by changes to the composition of emissions?
\item How does emissions spatial heterogeneity alter the activity of CCN?
\end{itemize}

\item Development of a framework for evaluating spatial heterogeneity impact on aerosol properties, including CCN activity
\begin{itemize}
\item Transport vs. aerosol treatment graph 
\item Future extension could quantify the structural uncertainty for the representation of aerosols (properties, ccn activity, etc.) in coarser resolved models such as regional and global scale models that use modal/sectional aerosol treatments and parameterize turbulence 
\end{itemize}

\end{itemize}

\begin{figure}[h]
	\centering
	\includegraphics[width=\textwidth]{particle-resolved-model.pdf}
	\caption{Cartoon representation of a particle resolved model}
	\label{fig:test}
\end{figure}

\begin{table}[h!]
\centering
\begin{tabular}{||c c c c||} 
 \hline
 Col1 & Col2 & Col2 & Col3 \\ [0.5ex] 
 \hline\hline
 1 & 6 & 87837 & 787 \\ 
 2 & 7 & 78 & 5415 \\
 3 & 545 & 778 & 7507 \\
 4 & 545 & 18744 & 7560 \\
 5 & 88 & 788 & 6344 \\ [1ex] 
 \hline
\end{tabular}
\caption{Table to test captions and labels.}
\label{table:1}
\end{table}