% !TEX root = ./main.tex
\chapter{Introduction}

This chapter discusses key background motivating this thesis, including fundamental properties of aerosols, their feedbacks on the climate, and the role of emissions spatial heterogeneity in altering the aerosol state. A description of the coupling between emissions spatial heterogeneity and aerosol processes is provided, and a brief review of current efforts to quantify and parameterize the sub-grid scale variability of aerosols is presented. Subsequently, the representation of aerosols in modeling frameworks is discussed and the current state of high-resolution coupled aerosol-transport models capable of resolving emissions spatial heterogeneity is reviewed. Finally, existing gaps in the literature are outlined and objectives are presented to answer primary avenues of inquiry in this thesis. 

\section{The complexity of atmospheric aerosols}\label{aerosol_properties}

An aerosol is a collection of particles composed of one or more chemical species that are suspended in a fluid or gas. In the atmosphere, aerosol particles vary considerably in terms of their physical properties such as size, composition, and origin. Additionally, the chemical, thermodynamic, and radiative properties of aerosol particles can alter the state of the aerosol and the surrounding environment through numerous feedback mechanisms. In turn, aerosol particles exhibit a complex, non-linear coupling with the environment that spans broad spatial and temporal scales.

\subsection{Particle size}

\begin{figure}[!t]
	\centering
	\includegraphics[width=.75\textwidth]{chapter1/SP_Figure_8-11.pdf}
	\caption{Common aerosol number and volume distributions organized by mode. Taken from \textcite{seinfeld_atmospheric_1998} with permission.}
	\label{fig:size_dists}
\end{figure} 

Aerosol particles are typically measured by their diameter where spherical morphology is assumed. The smallest particles have diameters on the order of 1~nm and are produced via the nucleation of low-volatility vapors. On the opposite extreme of particle sizes, the largest particle diameters can exceed 100~$\upmu$m. In total, aerosols span approximately five orders of magnitude \parencite{seinfeld_atmospheric_1998}. To capture the broad scale of particle diameters that may be present in a population of aerosol particles, aerosol size distributions often represent the number concentration of particles as a function of the logarithm of particle diameter. The particle size distribution may be represented by multiple modes---lognormal size distributions---that are differentiated by the characteristics of particles within each mode, including growth and removal mechanisms. Typically, three distinct modes are present in a particle size distribution: the nucleation, accumulation, and coarse mode. Figure \ref{fig:size_dists} shows typical size distributions containing multiple overlapping lognormal modes. 

\subsection{Production and removal mechanisms}
Nucleation mode particles are up to 20~nm in diameter and undergo rapid growth as gas-phase species condense onto the particle surface or as particles inelastically collide through coagulation. They are removed from the nucleation mode by growth into the accumulation mode, which spans particle diameters from 0.1~$\upmu$m to 2~$\upmu$m. In addition to particles that enter the accumulation mode through growth by condensation or coagulation, particles may be released directly into the accumulation mode via primary emissions. Removal mechanisms such as dry deposition are least efficient in the accumulation mode, allowing particles to remain suspended in the atmosphere for days to weeks. Particles in the coarse mode have diameters exceeding 2~$\upmu$m  and are produced by mechanical processes such as abrasion and the resuspension of dust. Due to their size, particles in the coarse mode are rapidly removed by gravitational settling within minutes to hours \parencite{seinfeld_atmospheric_1998}. This multi-modal description of the aerosol size distribution points to the inherent complexity of aerosol population dynamics---production, growth, and removal mechanisms differ considerably by particle size. 

\subsection{Aerosol compositional diversity}
As noted, production mechanisms vary across aerosol modes (e.g., nucleation of low-volatility vapors, emission of primary aerosol into the accumulation mode, resuspension of coarse particles, etc.). These processes involve numerous chemical species. For example, whereas volatile organic compounds (VOCs) such as isoprene and other organic carbon (OC) species may undergo oxidation reactions which lower their volatility and promote particle nucleation, particles released directly into the accumulation or coarse mode as primary aerosol may consist of combustion by-products such as black carbon (BC) or naturally occurring aerosol such as sea salt spray and mineral dust. Furthermore, gas-particle partitioning allows inorganic gas phase species such as H$_2$SO$_4$, HNO$_3$, and NH$_3$ to enter the aerosol phase. Both the release of primary aerosol and the partitioning of compounds into the aerosol phase via gas-particle partitioning alter the properties of the aerosol population including hygroscopicity and optical properties which are key to the feedbacks between aerosols and climate.  

%\hl{Here its worth acknowledging contribution of precursor emissions to chemical aging, secondary production of aerosol-phase matter, changes to aerosol mixing state, etc.}. As a result, aerosol particles are compositionally diverse. 

%In addition to diversity in the composition of aerosol particles across the size distribution, aerosol populations also exhibit spatiotemporal variations which alter the local structure and composition of the aerosol. The geographic distribution of emission sources, varied land use, and topography lead to spatial heterogeneities in the emission of gas-phase precursors and primary aerosols. Additionally, temporal trends alter the meteorological state of the atmosphere and the concentration of reactive gas or aerosol-phase species. For instance, diurnal variation in the structure of the boundary layer due to surface heating determines the strength of vertical transport and mixing of primary aerosol or reactive gas-phase species. \hl{[Could talk about photolysis]}. Furthermore, the timing of emissions may play a crucial role in determining whether a chemical reaction will take place; reactive species must be present in the same space and time to undergo reaction.



\subsection{Impacts of aerosols on climate}

Aerosols alter the Earth's radiative budget directly through scattering and absorption of shortwave (solar) radiation. The scattering of solar radiation by aerosols back out to space increases planetary albedo, thereby decreasing the intensity of radiation reaching the Earth's surface \parencite{charlson_climate_1969, charlson_climate_1992}. As a result, scattering generally contributes a net cooling effect. The intensity of scattering depends on the composition of the aerosol, with strongly-scattering species including sulfate and nitrate. Aerosols may also absorb broadband radiation, re-emitting in the form of thermal radiation that results in a net warming effect. Absorption varies by aerosol species; strongly absorbing species include carbonaceous aerosol such as black carbon and brown carbon.  Scattering and absorption of solar radiation due to aerosols alters the stability of the atmosphere due to changes in the vertical profile of temperature \parencite{li_scattering_2022, lau_observational_2006}. 

In addition to direct aerosol-radiative effects, aerosols also alter the climate through indirect effects with clouds commonly referred to as aerosol-cloud interactions. Hygroscopic aerosol particles act as cloud condensation nuclei (CCN), thereby allowing water vapor to condense onto their surface at ambient supersaturations $S$ typical of the troposphere ($S\lesssim1\%$). \textcite{twomey_influence_1977} was the first to note that higher concentrations of CCN result in a greater abundance of small cloud droplets. This in turn leads to an increase in cloud albedo, causing greater reflection of solar radiation back to space and thus a net cooling effect on climate. In addition to the Twomey effect, the impact of aerosol number concentration on droplet size can delay or prevent the onset of collision-coalescence necessary to initiate precipitation. This effect was first discovered by \textcite{albrecht_aerosols_1989} and enhances the lifetime of clouds, thereby prolonging the reflection of solar radiation. 


% Need $$ environment for the ERFs (negative signs are incorrect!
The global mean effective radiative forcing (ERF) due to the combination of direct and indirect effects is estimated by the Intergovernmental Panel on Climate Change (IPCC) to be in the range of $-2.0$ to $-0.6$~\si{W.m^{-2}} within 95\% confidence, with a mean of $-1.3$~\si{W.m^{-2}} \parencite{ipcc_report_2021}. Separating the ERF into forcing due to aerosol-cloud interactions and direct radiative forcing, aerosol-cloud interactions contribute the largest magnitude of forcing in the range $-1.7$ to $-0.3$~\si{W.m^{-2}} with a mean of $-1.0$~\si{W.m^{-2}}. Direct effects contribute $-0.6$ to $0.0$~\si{W.m^{-2}} with a mean of $-0.3$~\si{W.m^{-2}}. 

The magnitude of uncertainty in ERF due to aerosol direct and indirect effects remains large due to a host of factors. As discussed in Section \ref{aerosol_properties}, aerosol particle size and composition are highly varied and determine climate-relevant properties including a particle's scattering and absorption coefficients and its hygroscopicity. Representing the full range of aerosol composition and properties in a modeling framework is highly computationally expensive and current state-of-the-science global scale climate models use simplified aerosol treatments such as sectional or modal models (aerosol model treatments are discussed in more detail in Section \ref{aerosol_model_treatments}). Furthermore, estimates for ERF due to aerosol-cloud interactions in particular are poorly constrained due to limited understanding of the coupling between microphysical phenomena and cloud macrophysical structure for deep convective clouds where phase transitions complicate the role of aerosols and thermodynamic feedbacks \parencite{fan_review_2016}. In addition, aerosols are highly spatially heterogeneous due to localized sources, resulting in varied concentrations and properties that determine the local activity of CCN and associated aerosol-cloud interactions.

\section{Spatial heterogeneity in the atmosphere and its impact on aerosols}

\subsection{Coupling between surface heterogeneities and atmospheric state}

There exists a well established link between surface spatial heterogeneities and their impacts on the evolution of the atmospheric state. For instance, \textcite{fast_impact_2019} conducted a joint observation and modeling study to evaluate the role of soil moisture heterogeneity in promoting deeply convecting clouds. The authors compared observations collected during the Holistic Interactions of Shallow Clouds, Aerosols, and Land-Ecosystems (HI-SCALE) campaign against a set of large-eddy simulations (LES) where the spatial heterogeneity of soil moisture was varied from a constant distribution to higher variability which closely matched the observed soil moisture spatial heterogeneity.  The authors found that under modeling scenarios with smoothly varying soil moisture, clouds did not develop into open-cell, deep convective cumulus capable of precipitating and instead were characterized by shallow, uniform non-precipitating clouds. In order to replicate the degree of cloud heterogeneity and the development of deeply convecting clouds observed during the HI-SCALE campaign, realistic spatial variability in the modeled soil moisture distribution was required. 

In addition to soil moisture fluxes, spatial heterogeneity in surface heat fluxes has been shown to be critical to the development of atmospheric circulation. \textcite{lee_effect_2019} conducted an idealized LES study in which surface heat fluxes (including both sensible and latent heat flux) were prescribed by checkerboard patterns of ranging spatial heterogeneity (most heterogeneous being the lowest frequency checkerboard pattern with the largest pattern length scale, and the least heterogeneous being the highest frequency patterns with the smallest pattern length scale). The authors found that secondary circulation developed under scenarios with the highest spatial heterogeneity and minimal background winds (less than 2~\si{m.s^{-1}}). This circulation was responsible for transporting moisture from checkerboard regions with greater latent heat flux to drier regions with lesser latent heat flux. These secondary circulations develop via the same principle as sea-breeze circulations and is due to the difference in sensible heat flux at the surface between adjacent wet and dry regions. Dry regions are characterized by large sensible heat fluxes which promote convection, whereas wet patches result in greater latent heat flux. This results in a pressure gradient force between dry regions characterized by convergence at the surface and corresponding subsidence in wet regions. Such circulations are referred to as nonclassical mesoscale circulations and are described thoroughly by \textcite{segal_nonclassical_1992}.

These studies illustrate a complex coupling between surface heterogeneities and the atmospheric state. Importantly, this coupling gives rise to changes in atmospheric properties downstream of the direct interaction between surface heterogeneities and the atmosphere such as modifications to cloud type and the development of secondary circulation.   

\subsection{Coupling between emissions spatial heterogeneity and aerosol processes}\label{couple-emiss-sh-aerosol-process}

The spatial distribution of primary aerosol emissions and precursor gas phase emissions results in spatially varying concentrations that span orders of magnitude and complex variability in the composition of aerosols. For example, urban aerosol number concentrations are highly variable; whereas a significant number of nucleation mode particles ($\sim10^5$--$10^6$~\si{cm^{-3}}) may be found nearby busy highways, the concentration of nucleation mode particles is significantly reduced downwind of the highway due in large part to coagulation \parencite{zhu_study_2002}. By comparison, rural aerosol concentrations are more spatially uniform and lower in number with concentrations ranging between $\sim10^3$--$10^4$~\si{cm^{-3}}. Whereas urban aerosol are composed of a mixture of primary carbonaceous aerosol released from vehicular and industrial combustion and species resulting from gas-particle partitioning of emitted gas phase compounds such as NO$_x$ or SO$_2$, rural aerosol contain a large fraction of organics resulting from the oxidation of biogenic volatile organic compounds (BVOCs) in the gas phase to form secondary organic aerosol (SOA). In rural regions with significant amounts of agricultural land use, ammonium may also be abundant \parencite{seinfeld_atmospheric_1998}. 

The spatial heterogeneity of both gas phase and aerosol number concentrations impacts how particles age due to concentration dependent processes such as coagulation and gas-particle partitioning. For a number distribution $n(v,t)$ that is a function of particle volume $v$ and time $t$, the rate of change to the number distribution due to coagulation is defined as 
\begin{equation}
\frac{\partial n(v, t)}{\partial t} = \frac{1}{2}\int_0^{v}K(v-v', v')n(v-v', t)n(v', t)dv' - n(v,t)\int_0^{\infty}K(v',v)n(v',t)dv',
\label{eq:coag}
\end{equation}
where $K(v_1, v_2)$ is the coagulation kernel between particles of volume $v_1$ and $v_2$. The first term on the right hand side of Equation \ref{eq:coag} is coagulation gain while the second term is coagulation loss. Note how each term is proportional to the square of the number distribution. This causes the rate of coagulation to be highly sensitive to changes in aerosol number concentration, whereby highly polluted regions (such as nearby highway emissions) experience elevated rates of coagulation. 

In addition to coagulation, the rate of chemical reactions in both the gas phase and gas-particle partitioning are concentration dependent and thus the spatial heterogeneity of emitted compounds  determines the effective rate at which such reactions proceed. An extensive body of literature evaluates the effects of chemical segregation (i.e., the degree to which precursor compounds are spatially separated or collocated) on the abundance of reaction products in the atmospheric boundary layer \parencite{schumann_large-eddy_1989, sykes_turbulent_1994, molemaker_control_1998, krol_effects_2000, vinuesa_fluxes_2003, auger_chemical_2007, pugh_influence_2011, ouwersloot_segregation_2011, dlugi_balances_2014, kim_impact_2016, li_error_2021, wang_segregation_2022}. All of these studies focus on second order gas phase reactions which are prevalent in atmospheric chemistry and utilize LES to resolve turbulence-chemistry interactions. Initial studies focused on generic species and a range of imposed reaction rates \parencite{schumann_large-eddy_1989, sykes_turbulent_1994, molemaker_control_1998}. Subsequently, modeling studies have investigated the production and destruction of ozone and oxidation of generic VOCs \parencite{krol_effects_2000, auger_chemical_2007} and more recently oxidation of isoprene by OH \parencite{pugh_influence_2011, ouwersloot_segregation_2011, dlugi_balances_2014, kim_impact_2016}. Advances in computing have allowed the use of direct numerical simulations of gas phase reactions in the planetary boundary layer \parencite{li_error_2021} and the modeling of entire urban regions with LES to evaluate chemical segregation \parencite{wang_segregation_2022}. Note that these studies do not model aerosols, however the coupling between the gas phase and aerosols through gas-particle partitioning suggests chemical segregation due to the spatial heterogeneity of emissions likely influences the aerosol state. %including composition and climate relevant properties such as optical properties and CCN activity. 

\subsection{Sub-grid scale variability of aerosols}

The spatial heterogeneity of emission sources and aerosol processes such as coagulation and gas-particle partitioning vary on scales smaller than the grid resolution of current global climate models (GCMs) and regional scale models. This further complicates calculation of climate relevant properties  (optical properties and CCN activity) and their associated ERF due to uncertainty in the sub-grid variability of both gas phase and aerosol concentrations and properties. Past efforts to quantify the sub-grid variability of aerosols and their associated properties have centered around the comparison of coarse resolution, large-scale observational or modeling domains against higher resolution versions of the same domain. The resulting difference in the aerosol state between coarse and fine resolution domains serves as a measure of uncertainty in coarse-resolved models due to the inability to capture the full spatial heterogeneity of aerosols. Such uncertainty results from the structural representation of physical processes within the modeling framework. Modelers must make structural choices which define how the modeling framework represents physical phenomena. This includes, for example, the aerosol treatment, the choice of chemical mechanism, or the representation of dynamics. These processes are represented via mathematical frameworks which often necessitate simplifications (e.g., reducing the complexity of a physical or chemical mechanism) or assumptions (e.g., ignoring certain scales or processes). Thus, a mathematical model will not perfectly match the real-world processes or dynamics it seeks to represent due to the choice of modeling simplifications or assumptions. This gives way to uncertainty due to the structural characteristics of the model which we refer to as structural uncertainty. Notably, structural uncertainty in a models’ predicted quantities such as CCN activity result from a complex coupling of all structural choices; the resulting difference between model predictions and nature quantifies structural uncertainty due to the aggregate, often nonlinear, interplay of a model’s structural choices. Note that structural uncertainty differs from uncertainty due to the choice of model parameters (i.e., parametric uncertainty). 

\textcite{lin_quantification_2017} conducted a modeling study to evaluate the sub-grid variability in aerosol number and mass concentrations over the southern Pacific Ocean. The model, WRF-Chem, was run over a $900\times900$~\si{km^2} region at a grid resolution of $3\times3$~\si{km^2} and aerosols were modeled using the three-mode MAM3 scheme. A $360\times360$~\si{km^2} study region in the center of the modeling domain was further divided into various grid boxes representative of GCM resolutions ranging from $180\times180$~\si{km^2} to $30\times30$~\si{km^2}. At each resolution, grid cell averages were computed alongside the sub-grid standard deviation at the native model resolution of $3\times3$~\si{km^2}. The authors found that aerosol number and mass concentrations are highly variable, with the greatest variability in standard deviation found in the free troposphere. 

\textcite{weigum_effect_2016} quantifed sub-grid variability in aerosol optical depth (AOD) and CCN concentrations for a modeling region encompassing the United Kingdom and northern France. WRF-Chem modeling runs were conducted initially at 10~km resolution and subsequently at coarser resolutions (40, 80, 160~km). Aerosols were modeled using a three-mode version of the MADE/SORGAM module, which combines the Modal Aerosol Dynamics model for Europe (MADE) with the Secondary Organic Aerosol Module (SORGAM). When comparing the 80~km resolution case (typical of most GCM resolutions) against results at 10~km, the authors found an underestimation of AOD by 20-40\% and an underestimation of CCN concentrations by 33\% on average\footnote{Note that \textcite{weigum_effect_2016} compared coarse resolution results against the highest resolution scenario for two types of simulations: runs where the resolution of only the aerosols and gasses were lowered (all other environmental variables and dynamics were resolved at the base 10~km resolution), and those in which all model parameters and dynamics were represented on the coarse grid mesh. Results discussed here compare the high and low resolution simulations where the resolution of all model parameters was lowered (these simulations are referred to as ``FRA10" for the full-resolution 10~km run and ``FRA80" for the full-resolution 80~km run). This approach matches the manner in which past studies have evaluated sub-grid variability across modeling scales and crucially considers the coupled impact of resolution on meteorology and aerosol processes.}. They noted that the processes most affected by neglecting aerosol sub-grid variability include gas-phase chemistry and aerosol water uptake. For instance, changes in AOD are linked to the water content of the accumulation mode, which is largely regulated by gas-particle partitioning of the sulfate-nitrate-ammonium system. The authors noted that boundary layer nitrate concentrations are up to 20\% lower in the 80~km scenario, leading to a reduction in aerosol water content. The impact of aerosol sub-grid variability on nitrate concentrations is particularly meaningful, as recently GCMs have begun to include nitrate aerosol in the calculation of direct radiative forcing.

\textcite{qian_investigation_2010} conducted a modeling study to measure sub-grid variability of both gasses and aerosols in a region over central Mexico. The authors used WRF-Chem and compared modeling results at $75\times75$~\si{km^2} resolution against two higher resolution scenarios ($15\times15$~\si{km^2} and $3\times3$~\si{km^2}). Aerosols were modeled using the 8-bin sectional MOSAIC model and CBMZ was used for gas phase chemistry. Probability density functions (PDFs) were created for the distribution of trace gases and aerosols captured by the higher resolution simulations over a region of high urban emissions (Mexico City) and indicated that longer-lived compounds (e.g., CO in the gas phase and BC in the aerosol phase) tended to have broader distributions, indicating greater sub-grid variability. Faster reacting species (e.g., ozone in the gas phase and sulfate, nitrate, and ammonium in the aerosol phase) tended to have narrower PDFs, suggesting less sub-grid variability. The daytime vertical profile of sub-grid variability for trace gasses including CO and ozone were nearly uniform within the planetary boundary layer, indicating they were well mixed, whereas the sub-grid variability of BC, sulfur, nitrate, and ammonium reached a maximum at the top of the planetary boundary layer. Emissions contributed significantly to sub-grid variability by up to 50\%, especially during the daytime for less reactive species in the vicinity of urban regions. 

\textcite{gustafson_jr_downscaling_2011} extended on the work of \textcite{qian_investigation_2010} by using the same modeling region and simulation setup; however, their analysis focused on the contribution of sub-grid variability to direct aerosol radiative forcing. The authors found that over the Mexico City metropolitan area, daytime mean bias for top-of-atmosphere direct aerosol radiative forcing was in excess of 30\% when comparing modeling results at coarse resolution ($75\times75$~\si{km^2}) against the highest resolution scenario ($3\times3$~\si{km^2}). Furthermore, the depiction of emissions contributed significantly to direct aerosol radiative forcing due to the dependence of emissions rates such as dust on local wind speeds. Additionally, the authored noted that higher resolution simulations better resolved local flow heterogeneities that resulted in a greater concentration of suspended dust. 

\textcite{crippa_impact_2017} conducted a modeling study over eastern North America to measure the effects of resolution of meteorological, gas phase, and aerosol properties including AOD. WRF-Chem was used alongside the three-mode MADE/SORGAM module for representation of aerosols. Simulations were run at both 60 and 12~km resolution. In addition to direct comparison between model runs at each resolution, meteorological outputs were evaluated against reanalysis data while simulated AOD was compared against MODIS satellite observations. The skill of model outputs was measured using Brier skill scores. The authors found that the higher resolution 12~km simulations agreed more closely with meteorological reanalysis and AOD observations, however, notable differences were still present at 12~km, especially in comparing AOD measurements to MODIS observations. The authors note that this discrepancy may be in part due to the choice of a modal aerosol representation, as the geometric standard deviation of each mode is fixed in a modal model and past studies have shown that modeled AOD is sensitive to the choice of standard deviation \parencite{brock_aerosol_2016, mann_intercomparison_2012}. 

\textcite{fast_using_2022} evaluated the sub-grid scale variability of aerosol properties in an observational campaign using aircraft measurements over the Atmospheric Radiation Measurement (ARM) program's Southern Great Plains (SGP) site in north Oklahoma. A $162\times162$~\si{km^2} study region was divided into gridded domains representative of model resolutions typical of GCMs (81~km), future climate models (27~km), current global forecast models (9~km), and cloud-resolving models (3~km). Aircraft measurements of aerosol number distributions, CCN concentrations. and aerosol composition were averaged within each grid resolution and cell averages were compared against mean values within coarse-resolved 81~km cells. The authors found considerable sub-grid variability in the concentration of aerosol organic matter which comprises much of the aerosol composition due to the abundance of biogenic sources that release precursor BVOCs in the vicinity of the SGP site. 3~km cell averaged size distributions were shown to have much higher variability than their 81~km cell averaged counterpart due to local industrial sources of ultrafine particles and indicated that bi-modal or multi-modal distributions were averaged out at coarse resolution. The authors concluded that differences in the representation of the size distribution due to spatial averaging of sub-grid variability may lead to errors in CCN concentrations for GCMs. 

\subsection{Modeling approaches and parameterizations for emissions sub-grid scale variability}

Numerous modeling approaches and parameterizations have been developed for incorporating the effects of sub-grid variability resulting from the spatial heterogeneity of both gas phase and aerosol emissions. A straightforward approach to representing sub-grid variability is simply to refine the mesh to sufficient resolution, however global modification to the grid resolution imposes significant computational cost, especially for 3-dimensional domains where a doubling of resolution along each dimension results in an eightfold increase in the total number of grid cells. Alternatively, adaptive grid modeling allows local refinement of the mesh in regions of high heterogeneity such as near localized emissions sources. The total number of grid cells remains the same under refinement, however this comes at the cost of coarser resolution in regions that are not subject to refinement. Adaptive grid modeling has been applied in numerous regional-scale air quality models in order to improve representation of emissions plume structure and spatial heterogeneity \parencite{karamchandani_sub-grid_2011}.  

While adaptive grid modeling lowers computational cost relative to the approach of global mesh refinement, computational overhead is incurred due to the need to recompute the grid mesh at regular intervals. By contrast, plume-in-grid (PinG) modeling preserves the original, Eulerian mesh resolution while improving representation of emissions plumes at sub-grid scales via the use of an embedded Lagrangian framework. The embedded model is used to track the local dispersion of sub-grid scale emissions plumes until local concentrations are sufficiently diffuse, at which point the plume is handed over to the Eulerian grid. PinG modeling requires a-priori knowledge of the plume morphology in order to accurately represent its dispersion, a fact which hampered early implementations of PinG due to the limited choice of plume geometries (e.g., ellipses, Gaussian distributions, etc.) \parencite{karamchandani_sub-grid_2011}. In recent years, improved plume models such as SCICHEM have been embedded in numerous chemical transport models, including the U.S. Environmental Protection Agency's CMAQ model, CAMx, and the Weather Research and Forecasting model coupled to chemistry (WRF-Chem). 

 \textcite{galmarini_modeling_2008} proposed a parameterization for the sub-grid variability of emitted, non-reactive scalars in the planetary boundary layer applicable to models ranging from regional to global scale. Their approach required representing the evolution of the scalar concentration variance via a prognostic equation, for which closure of the covariance between emission fluctuations and scalar concentration values was presented. Closure constants were derived via LES and the parameterization was evaluated by comparing Reynolds-averaged Navier-Stokes simulations with the sub-grid parameterization against LES results. 
 
\textcite{cassiani_stochastic_2010} developed a method for representing the emission, transport, and dispersion of reactive scalars (e.g., reactive gas phase compounds) via an ensemble of stochastic fields. The stochastic fields represent the concentration PDF, where the transport equation solution for the concentration PDF is solved via the ensemble of stochastic field members. Provided an emissions inventory with resolution higher than that of the modeling domain, one can construct an emissions PDF which provides a source term for the evolution of the stochastic fields. By contrast to the method of \textcite{galmarini_modeling_2008}, the stochastic fields method is notable in providing formal closure for an arbitrary number and type of chemical reactions. This aspect is particularly valuable for atmospheric chemistry models given the impacts of chemical segregation due to emissions heterogeneity and turbulence on second order reactions as discussed previously.   

%\cite{valari_transferring_2010}
In addition to modeling approaches and parameterizations which address the sub-grid scale variability of emitted scalars such as gas phase species, recent progress has been made in the sub-grid scale representation of aerosol processes such as coagulation. \textcite{pierce_parameterization_2009} developed a parameterization for the coagulation of aerosol particles within sub-grid scale plumes. The parameterization estimates the number of particles that remain after coagulation and which are transferred outside the emitting grid cell. Subsequently, \textcite{sakamoto_evolution_2016} created a parameterization for growth in aerosol diameter due to sub-grid coagulation in biomass burning plumes. \textcite{ramnarine_effects_2019} applied the sub-grid coagulation parameterization of \textcite{sakamoto_evolution_2016} to evaluate its impact on global aerosol size distributions and radiative effects in the GEOS-Chem-TOMAS (TwO-Moment Aerosol Sectional) model. The authors found that including sub-grid coagulation led to a 37\% reduction in particles 80 nm and larger (a chosen proxy for particles capable of activating as CCN), resulting in a decrease in the magnitude of indirect radiative forcing ($-76$ to $-43$~\si{mW.m^{-2}}). Furthermore, the inclusion of sub-grid coagulation shifted the size distribution to more efficient scattering, increasing the all-sky direct radiative effect from $-224$ to $-231$~\si{mW.m^{-2}}. It is notable that the inclusion of sub-grid coagulation altered the magnitude of indirect radiative forcing by approximately 40\% and that this change is entirely due to the improved representation of a single aerosol process. This suggests that models which resolve additional sub-grid scale aerosol processes such as gas-particle partitioning and their coupling with emissions spatial heterogeneity may considerably differ from coarser resolved models and those that do not represent sub-grid scale aerosol processes. 


\section{Representation of aerosols across modeling frameworks}\label{aerosol_model_treatments} 

\subsection{Aerosol modeling treatments}

We have shown that the size, composition, and properties of aerosols are highly diverse and contribute to the complexity of their environmental feedbacks including direct and indirect radiative forcing. In turn, modelers must decide how much of the underlying aerosol complexity to represent when conducting aerosol-aware simulations. As an underlying principal, modelers must balance computational expense with aerosol model complexity (and therefore aerosol representability) and these two attributes generally move in tandem (i.e., computational cost increases as the complexity and representability of the aerosol treatment increases). 

\begin{figure}[!t]
	\centering
	\includegraphics[width=\textwidth]{chapter1/aerosol-model-treatments.pdf}
	\caption{Illustrations of aerosol modeling treatments, each for the same aerosol population. The following representations are shown: (a) Bulk, (b) Modal, (c) 1-D sectional, (d) Particle-resolved.}
	\label{fig:aerosol-models}
\end{figure}

Thus we may consider an aerosol modeling hierarchy organized from aerosol model treatments with low complexity and computational cost to treatments that possess high complexity and representability alongside increased computational expense. Figure \ref{fig:aerosol-models} shows a series of four aerosol model treatments from across the aerosol modeling hierarchy with low complexity models in the top left and high complexity models in the bottom right. For each subfigure, the model treatment represents the same underlying aerosol population composed of three general species A, B, and C. Moving through this Figure \ref{fig:aerosol-models} sequentially, (a) corresponds to a bulk aerosol model in which only the bulk composition of the population is tracked. The aerosol population is not size-resolved, meaning that the population is monodisperse with a single prescribed particle diameter. Bulk models are the simplest aerosol treatment and require low computational cost. 

Figure \ref{fig:aerosol-models} subfigure (b) shows a modal aerosol treatment. In a modal model, the aerosol population is size-resolved and numerous lognormal distributions (``modes") are used to represent the size distribution. Within each mode, the aerosol composition is the same (i.e., the particles are internally mixed within a mode) and across modes the composition can differ (the aerosol particles are externally mixed across modes). Therefore, the compositional complexity of a modal model is tied to the number of modes represented by the model. The example shown illustrates a three-mode model, which is a common configuration for modal aerosol models. A thorough listing of modal models ranging in configuration from three to 16 modes is provided in \textcite{riemer_aerosol_2019} Table 5. Note that because modal models must maintain lognormal distribution shapes, this introduces structural uncertainty in the representation of aerosols as the size distribution is artificially altered to ensure lognormal distributions. As discussed previously, this limitation of modal models helped to explain discrepancies found by \textcite{crippa_impact_2017} between satellite observed AOD and modeled values.  

Moving to higher aerosol model complexity, Figure \ref{fig:aerosol-models} subfigure (c) shows a sectional aerosol treatment. Sectional models represent the aerosol size distribution using a series of adjacent bins (``sections"). Similar to the modal aerosol treatment, the composition of particles within each bin is the same while the composition is allowed to vary across bins. The example sectional treatment illustrates a 1-D sectional model with eight bins. Higher dimensional sectional models exist such as 2-D and 3-D representations whereby the additional dimension(s) is used to represent additional aerosol properties (e.g., the 3-D sectional framework of \textcite{ching_three-dimensional_2016} adds dimensions for BC mass fraction and hygroscopicity).

Lastly, Figure \ref{fig:aerosol-models} subfigure (d) shows a representation of a particle-resolved model. Among the aerosol modeling hierarchy, particle-resolved models possess the highest degree of compositional complexity and representability due to the direct representation of aerosol particles via a set of computational particles. The size and composition of each particle is allowed to vary and evolution of the aerosol state is tracked by solving the general dynamic equation describing processes (e.g., transport, coagulation, condensation, emission, deposition, nucleation) that alter and age the aerosol population. In a particle-resolved model, computational particles are represented by an $A$ dimensional vector $\vec{\mu^i}\in \mathbb{R}^A$ where $A$ is equal to the number of aerosol species and the vector magnitude along each dimension $\vec{\mu_a^i}$ corresponds to the mass of species $a$ in computational particle $i$ for $a=1,...,A$ and $i=1,...,N_p$. Changes to aerosol composition due to gas-particle partitioning directly alter the computational particle vector $\vec{\mu^i}$. Processes including coagulation, deposition, and dilution are treated in a stochastic manner building off the Monte Carlo algorithm for stochastic collision-coalescence of cloud drops developed by \textcite{gillespie_exact_1975}. While particle-resolved models represent individual computational particles, it is customary to assign each particle a weighting (i.e., a multiplicity factor) in order for the set of computational particles to accurately resemble the number concentration of typical aerosol populations. This is because the number of particles per unit volume can be quite high and representing each individual particle would be computationally prohibitive as the numerical cost of particle-resolved models tends to scale with the number of computational particles. 

Examples of particle-resolved models include the Particle Monte Carlo model (PartMC) \parencite{riemer_simulating_2009}. PartMC is a box model, meaning that the relative position of particles within a grid cell are not tracked. PartMC has been used to describe the complex mixing state (i.e., the compositional diversity within and across aerosols) of particles containing BC and their aging alongside quantification of the processes that contribute to aging \parencite{riemer_simulating_2009}. This represents a key advantage of particle-resolved models over coarser-resolved aerosol treatments, as the full complexity of the aerosol compositional diversity and the processes contributing to aging can be measured in a manner not possible with coarser-resolved aerosol models. Furthermore, particle-resolved models have been used to quantify error in coarser aerosol models regarding the prediction of climate-relevant aerosol properties including CCN activity and optical properties. \textcite{zaveri_particle-resolved_2010} developed a ``composition-averaging" post-processing technique for averaging the compositional diversity of particle-resolved output to match that of coarser-grained aerosol models such as sectional treatments. The authors found that when the aerosol population composition was averaged to match that of a 10-bin sectional model, the resulting CCN concentrations were overestimated in the sectional treatment by up to 125\%. This illustrates the large degree of structural uncertainty present in coarser-resolved aerosol models regarding the representation of CCN activity. \textcite{fierce_quantifying_2024} directly quantified this structural uncertainty between particle-resolved and modal estimates of CCN activity by comparing PartMC with the four-mode MAM4 model. The authors found that CCN activity diverged between the two models after a couple of hours and that conditions with rapid aging such as in polluted regions with a high rate of coagulation and gas-particle partitioning amplified the model disagreement. 

\subsection{Coupled aerosol-transport models}
Models which jointly represent aerosols, their aging, transport, and feedbacks on the environment alongside broader-scale atmospheric dynamics are typically a coupled system of models, one responsible for the aerosol representation as previously discussed and another model which handles atmospheric transport and evolution of meteorological variables. Global and regional scale models typically represent transport via Reynolds-averaged Navier-Stokes which models the mean flow and fully parameterizes turbulent transport. Current generation climate models represent aerosols with a modal or sectional treatment. For instance, the U.S. Department of Energy's Energy Exascale Earth System Model (E3SM) uses the 4-mode MAM4 model \parencite{golaz_doe_2022} while the National Center for Atmospheric Research has developed the Community Earth System Model (CESM) to utilize a 40-bin sectional aerosol treatment \parencite{tilmes_description_2023}. As discussed previously, the spatial heterogeneity of emissions and other atmospheric constituents vary on scales smaller than the resolution of large scale models and thus turbulence-chemistry interactions and the impacts of chemical segregation between reactive compounds is not resolved. Joining these considerations with the impact of aerosol modeling treatment on the representation on aerosol properties such as CCN activity, there exists a need for high resolution models which utilize detailed aerosol treatments coupled with transport schemes which explicitly resolve both steady mean state and turbulent flow.

Recently, turbulence-resolving models such as UCLALES-SALSA \parencite{tonttila_uclalessalsa_2017} and the Dutch Atmospheric Large Eddy Simulation (DALES) model \parencite{de_bruine_explicit_2019} have coupled LES transport schemes along aerosol model treatments. Both models have been used to investigate aerosol-cloud interactions and compare model outputs against field campaign measurements. Although these models have high-resolution transport schemes, they each possess relatively coarse-resolution aerosol treatments. For instance, UCLALES-SALSA uses a 10-bin sectional treatment while DALES implements a modified version of the seven-mode M7 model to allow two additional hydrometeor modes. To our knowledge, no modeling framework has yet to leverage a high-resolution particle resolved aerosol treatment alongside turbulence resolving transport models such as LES. 

\section{Objectives of this thesis}

Past research has established a clear link between the spatial heterogeneity of both gas phase and aerosol emissions, the processes by which the aerosol age, and their resulting climate-relevant properties including CCN activity which contribute to indirect radiative forcing. Simultaneously, there exists large uncertainty in the radiative forcing due to aerosols which results in part from the coarse representation of both aerosols and their transport in global scale models and their coupling with non-linear processes that occur at the sub-grid scale such as coagulation and turbulence-chemistry interactions. Past research has led to the development of detailed aerosol model treatments and transport representations, however there has yet to be a direct coupling between particle-resolved aerosol models and turbulence-resolving transport models for use in quantifying the effects of emissions spatial heterogeneity on the aerosol state and CCN activity. 

This thesis aims to address this gap in the literature by conducting a series of first-of-a-kind simulations using the coupled model WRF-PartMC-LES which allows particle-resolved large eddy simulations. The impacts of emissions spatial heterogeneity on the aerosol state including CCN activity are investigated by quantifying changes to aerosol size, number concentration, mass fraction, and hygroscopic properties across a variety of emissions scenarios. This thesis serves to answer the following scientific questions:
\begin{itemize}
\item What is the effect of emissions spatial heterogeneity on aerosol processes (e.g., coagulation, gas-particle partitioning)?
\item As emissions spatial heterogeneity alters the aerosol processes that occur, how does this change the aerosol properties (e.g., composition, concentration, hygroscopicity) and resulting CCN activity? 
\item How is the impact of spatial heterogeneity on aerosols modulated by changes to the composition of emissions?
\end{itemize}

In addition to the primary goals of this thesis, the development of a coupled particle-resolved large eddy simulation modeling framework presents a valuable benchmark against which the representation of aerosols in coarser-resolved models (either in transport treatment or aerosol treatment) can be compared.

\begin{figure}[!t]
	\centering
	\includegraphics[width=\textwidth]{chapter1/Transport-vs-aerosol-model-struc-uncert.pdf}
	\caption{Use of WRF-PartMC-LES as a benchmark against coarser resolved aerosol-transport models for quantifying structural uncertainty in aerosol properties.}
	\label{fig:transport-vs-aerosol-model}
\end{figure} 

To illustrate this future role for WRF-PartMC-LES, Figure \ref{fig:transport-vs-aerosol-model} shows a representation of models in a coupled aerosol-transport model hierarchy. Models closer to the top right have high resolution in both space and aerosol treatment (here the spatial resolution is treated as a proxy for transport as higher resolution allows explicit depiction of turbulent transport scales). WRF-PartMC-LES serves as a benchmark against which coarser resolved models such as GCMs and regional scale models are evaluated. Structural uncertainty in key climate-relevant aerosol properties including CCN activity and aerosol optical properties could be quantified in a self-consistent manner, thereby improving understanding of uncertainty in direct and indirect radiative forcing due to aerosols.   

As noted previously, structural uncertainty in a model’s predicted quantities such as CCN activity results from the complex, often nonlinear, coupling of multiple structural choices. Thus difference in model predictions between, say, a GCM and WRF-PartMC-LES will capture the combined effect of all model components that are simplified in the GCM compared to WRF-PartMC-LES (e.g., impact of spatial resolution, transport treatment, aerosol representation). In addition to quantifying the coupled structural uncertainty in predicted quantities via separate aerosol-transport models,

Note that the structural uncertainty of aerosol-transport models is due to both the differing representations of the aerosol and transport across spatial scales. Let $\Sigma_{\phi}^{\text{GCM}}$ represent structural uncertainty in a quantity $\phi$ for a GCM. $\Sigma_{\phi}^{\text{GCM}}$ is the coupled structural uncertainty due to multiple differing model treatments within a GCM when compared against the benchmark model, WRF-PartMC-LES. Assume that the GCM and WRF-PartMC-LES utilize functionally similar sub-models for all other processes (e.g., chemical mechanism, radiation, etc.) besides the aerosol representation and the transport treatment. Thus, $\Sigma_{\phi}^{\text{GCM}}$ is a non-linear coupling of structural uncertainty due to both the aerosol representation and transport, and can be represented as

\begin{equation}
    \Sigma^{\text{GCM}}_{\phi} = \Sigma^{\text{aero}}_{\phi}\circ \Sigma^{\text{trans}}_{\phi}.
\end{equation}

Structural uncertainty in $\phi$ resulting from the aerosol representation will be due to the different aerosol treatment utilized by the GCM (typically a modal model) and the benchmark model (WRF-PartMC-LES utilizes a particle-resolved aerosol treatment). The structural uncertainty due to the aerosol treatment is then

\begin{equation}
    \Sigma^{\text{aero}}_{\phi} = \braket{\text{modal},\text{particle-resolved}}.
\end{equation}


We may estimate $\Sigma^{\text{aero}}_{\phi}$ by running a set of simulations, first with WRF-PartMC-LES followed by WRF-LES configured with a modal aerosol model treatment equivalent to the aerosol treatment in the GCM. Note that the transport treatment of these simulations remains the same. The difference\footnote{The use of a generic inner product signifies some degree of generality in how one may quantify the difference in $\phi$ due to the use of different aerosol treatments.} in a quantity $\phi$ between these two simulations provides an estimate of structural uncertainty due to the aerosol treatment.

Similarly, structural uncertainty in $\phi$ due to the transport treatment results from the use of Reynolds Averaged Navier-Stokes (RANS) in a GCM framework which fully parameterizes turbulence whereas WRF-PartMC-LES uses large-eddy simulations to explicitly resolve the largest scales of turbulent transport. The structural uncertainty due to the transport treatment is then

\begin{equation}
    \Sigma^{\text{trans}}_{\phi} = \braket{\text{RANS},\text{LES}}.
\end{equation}

In a manner similar to the estimate of $\Sigma^{\text{aero}}_{\phi}$, we may estimate $\Sigma^{\text{trans}}_{\phi}$ via a set of simulations in which the aerosol treatment is kept the same (i.e., particle-resolved), however, the transport representation is varied. This could be accomplished by first running a WRF-PartMC-LES simulation followed by a WRF-PartMC simulation in which the transport treatment is RANS. The difference in a quantity $\phi$ between these two simulations provides an estimate of structural uncertainty due to the transport treatment.

Note that $\Sigma^{\text{GCM}}_{\phi}$ is \textit{not} a linear combination of $\Sigma^{\text{aero}}_{\phi}$ and $\Sigma^{\text{trans}}_{\phi}$, as the coupling between the aerosol treatment and transport may introduce non-linear effects which are not present when estimating the structural uncertainty in the aerosol and transport treatments separately. 

To summarize, one may aim to constrain structural uncertainty in a quantity $\phi$ by conducting a series of comparison simulations: First, quantify the structural uncertainty directly for the model of interest (e.g., a GCM) by comparing output against the benchmark model WRF-PartMC-LES. Next, one may wish to evaluate the contribution to structural uncertainty resulting from differing aerosol treatments and transport representations between the model of interest and the benchmark model as discussed previously via a series of proxy model runs. To estimate structural uncertainty due to the aerosol treatment, compare WRF-PartMC-LES with WRF-LES configured with a modal aerosol treatment. To estimate structural uncertainty due to the transport representation, compare WRF-PartMC-LES with WRF-PartMC which utilizes RANS transport scheme. Note however that while WRF-PartMC-LES provides a useful means to \textit{estimate} structural uncertainty in coarser resolved models, WRF-PartMC-LES itself exhibits structural uncertainty due to various simplifying assumptions, such as parameterizing turbulent motion smaller than what is explicitly resolved on the domain, simplifications of the complex chemical mechanism governing gas-particle partitioning, and stochastic noise resulting from the Monte Carlo representation of aerosol particles in PartMC.   


