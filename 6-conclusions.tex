% !TEX root = ./main.tex
\chapter{Conclusions}

\section{Overview}
The spatial heterogeneity of both gas phase and primary aerosol emissions impact particle aging through non-linear, concentration-dependent processes including gas-particle partitioning and coagulation. Such aging alters aerosol properties relevant to climate including hygroscopicity which governs CCN activity. The coupling between emissions spatial heterogeneity through concentration-dependent processes to climate-relevant properties including CCN activity is of particular importance to improving the representation of aerosol radiative forcing in GCMs. The current generation of GCMs do not resolve the spatial heterogeneity of emissions, thereby representing emissions as uniform and dilute within grid cells on the order of $100x100$~\si{km^2}. 

Past studies which evaluate the impact of emissions and aerosol spatial heterogeneity at scales below that of typical GCM resolution employ simplified aerosol treatments (e.g., modal or sectional aerosol models) and do not explicitly resolve turbulence. The use of simplified aerosol treatments has been shown to result in disagreements in CCN activity as predicted by particle-resolved models, which depict the full compositional complexity of the aerosol. Simultaneously, the chemical segregation of spatially heterogeneous emissions has been shown to modify the effective rate of chemical reactions in the atmosphere and is highly dependent on turbulent mixing of precursor species. Past work points to the need for a modeling framework which couples a high-resolution aerosol model with a transport treatment which explicitly resolves turbulence such as large-eddy simulations (LES). While aerosol-aware LES models exist, no such framework couples a particle-resolved aerosol model with LES and former studies have not leveraged the utility of high resolution joint aerosol-transport models in investigating the effects of emissions spatial heterogeneity on climate-relevant aerosol properties. 

\subsection{Contributions}
Here, we investigate the impact of emissions spatial heterogeneity on both gas phase reactions and aerosol properties including CCN activity by conducting a series of particle-resolved large-eddy simulations. Simulations are characterized by the spatial heterogeneity of emissions and range from a uniform base case representative of a coarser-resolved model to the most spatially heterogeneous case in which all emissions are released within a single central grid cell of the high-resolution modeling domain. Emissions include both gas phase species and primary aerosol typical of urban regions. 

Analysis of the impacts of emissions spatial heterogeneity on the gas and aerosol phase requires quantification of the underlying heterogeneity; however, we show that past efforts in developing metrics which quantify spatial heterogeneity are poorly suited to the study of spatially varying quantities in the atmosphere, necessitating the development of a new metric for quantifying spatial heterogeneity. We present a novel metric developed in interdisciplinary collaboration with research colleagues for use in quantifying spatial heterogeneity and suitable across a range of applications. A numerically efficient Monte Carlo sampling method is presented which allows considerable speedup over a loop-based approach while maintaining a high degree of accuracy. 

\section{Findings and implications}

We find that emissions spatial heterogeneity has impacts on both the gas and aerosol phase. Under high spatial heterogeneity scenarios, ozone concentrations in the planetary boundary layer (PBL) are reduced by up to 12\% when compared to ozone levels in a scenario with uniform, dilute emissions. This suggests that coarser-resolved models may overestimate the oxidative potential of the PBL near sources characterized by high emissions spatial heterogeneity. 

Key aerosol processes including gas-particle partitioning and coagulation are impacted by emissions spatial heterogeneity, as we find significant changes to the sulfate-nitrate-ammonium system and an increased rate of coagulation under high emissions spatial heterogeneity scenarios. 

Changes to aerosol processes have downstream effects on CCN activity. Furthermore, modifications by emissions spatial heterogeneity to coagulation and gas-particle partitioning result in competing effects on the concentration of CCN at a given supersaturation level. Coagulation removes smaller particles that activate at high supersaturations, resulting in a decrease in CCN activity at high supersaturations for scenarios with high emissions spatial heterogeneity. Conversely, coagulation is not as efficient at removing larger particles that activate at lower supersaturations and gas-particle partitioning results in an increase of highly hygroscopic compounds such as ammonium nitrate under high emissions spatial heterogeneity. As a result, CCN activity at lower supersaturations increases with increasing spatial heterogeneity.  

%These results have important implications for large-scale modeling efforts such as GCMs that do not resolve 

\section{Study limitations and future work}

In this study, all emissions are released at the ground level. This is a simplification as some emission sources such as stack emissions from industrial sources are emitted at some height above the surface. Because the atmospheric stability often differs between the surface level and the planetary boundary layer, one may expect the shape and structure of the plume to differ if emissions are released directly into the boundary layer. For example, consider a nocturnal boundary layer/surface layer. If the emissions are released directly into the surface layer capped by a strong nocturnal inversion then emissions in the surface layer will be trapped and concentrations will be very high. If emissions are released from a plume above the surface layer, the absence of vertical mixing will lead to large concentration gradients between the region near the emissions plume and the surrounding ambient conditions. 

Here we chose spatial and temporal discretization a priori, mainly on the basis of what is feasible given our computational resources. A limitation of this approach is that we are uncertain whether we are adequately resolving all spatial and temporal scales at which gas phase and aerosol chemical reactions occur. Future work should quantitatively determine the necessary spatial and temporal scales for resolving relevant atmospheric chemistry reactions via metrics such as the Damköhler number.

This study uses one category of gas phase and aerosol emissions typical of urban settings. In actuality, emissions vary by land-use category and the interactions between emissions of different composition and spatial distribution play an important role in impacting atmospheric chemistry and aerosol processes. For example, consider NO$_x$ rich emissions resulting from vehicular combustion or industrial processes in an urban area. Adjacent to the urban region, say there exists a region dominated by agricultural emissions of NH$_3$. Without transport of NH$_3$ from the agricultural region, the aerosol in the urban region will be ammonium-poor resulting in little nitrate present. Across the boundary of these regions, one may expect a higher presence of nitrate in the aerosol phase due to the availability of ammonium. This illustrates the effects of chemical segregation between reactive species due to emissions in land-use categories that are spatially separated. Future work may wish to explore the role of emissions across multiple land-use categories in altering the aerosol state. 

As alluded to, the horizontal transport due to wind between emissions regions plays an important role in mixing reactive precursors and transporting aerosol particles. In this study, we neglect the influence of wind on emissions spatial heterogeneity. As wind generally plays a role in dispersing and diluting emissions plumes, one may expect wind to reduce the magnitude of trends found in this thesis, such as the increase in CCN activity at moderate supersaturations under high emissions spatial heterogeneity.

This thesis investigates the impact of emissions spatial heterogeneity on CCN activity. While this work has relevance for aerosol-cloud interactions and the indirect radiative forcing that results from changes to CCN concentrations, we do not explore direct radiative effects. One may wish to extend this analysis to include aerosol optical properties which govern aerosol direct effects, including changes to aerosol optical depth and layer heating due to absorption.

Through the use of particle-resolved modeling in a large-eddy simulation framework, this thesis establishes a benchmark for high resolution representation of aerosol processes in the boundary layer. A natural extension would be to conduct simulations with similar setups but instead use coarser-resolved treatments for the aerosols (e.g., modal or sectional representations) as well as the transport and turbulence treatment (e.g., use of Reynolds Averaged Navier Stokes modeling). This would allow intercomparison with the high-resolution results presented in this thesis and quantify structural uncertainties in desired quantities (such as CCN activity or aerosol optical properties) introduced by the use of coarser-resolved modeling treatments.
 