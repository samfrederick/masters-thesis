% !TEX root = ./main.tex
\chapter{Conclusions}

\section{Study limitations and future work}
\begin{itemize}
\item In this study, all emissions are released at the ground level. This is a simplification as some emission sources such as stack emissions from industrial sources are emitted at some height above the surface. Because the atmospheric stability often differs between the surface level and the planetary boundary layer, one may expect the shape and structure of the plume to differ if emissions are released directly into the boundary layer. For example, consider a nocturnal boundary layer/surface layer. If the emissions are released directly into the surface layer capped by a strong nocturnal inversion then emissions in the surface layer will be trapped and concentrations will be very high. If emissions are released from a plume above the surface layer, the absence of vertical mixing will lead to large concentration gradients between the region near the emissions plume and the surrounding ambient conditions. 

\hl{Is it possible that emissions released directly into the neutrally buoyant PBL wont be vertically mixed as strongly as they would if they were emitted from the surface? Reason why im not sure about this is because if the surface layer is unstable and the surface heat flux drives lots of convection then there will be plenty of mixing and vertical motion throughout the PBL.}
\end{itemize}