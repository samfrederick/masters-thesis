\begin{abstract}

% NOTE: abstract taken from CliMAS seminar... need to update!
Aerosol-cloud interactions remain a large source of uncertainty in climate models due to complex, nonlinear processes that alter aerosol properties and the inability to represent the full compositional complexity of aerosol populations within large-scale modeling frameworks. The spatial resolution of these models (typically 10-100 km) is often coarser than the spatially varying emissions in the modeled geographic region. This results in diffuse, uniform emission of primary aerosol and gas-phase species instead of spatially heterogeneous concentrations. Aerosol processes such as gas-particle partitioning and coagulation are concentration-dependent, and thus the representation of spatially heterogeneous emissions impacts aerosol aging and properties. This includes climate-relevant quantities key to aerosol-cloud interactions including particle hygroscopicity and cloud condensation nuclei (CCN) activity. 

This thesis investigates the impact of emissions spatial heterogeneity on CCN activity by using the particle-resolved model PartMC coupled to the Weather Research and Forecasting model configured for large-eddy simulations (LES). The resulting modeling framework, WRF-PartMC-LES, resolves turbulence-chemistry interactions and aerosol aging at the per-particle scale. The sensitivity of CCN activity in the planetary boundary layer (PBL) to emissions spatial heterogeneity is evaluated for primary aerosol and gas-phase emissions typical of urban regions. The pattern of emissions is varied to investigate a range of spatial heterogeneity scenarios. For each scenario, CCN activity is compared against a uniform emissions base case to determine the impact of spatially heterogeneous emissions. 

Spatial heterogeneity is quantified using a novel metric and a computationally efficient Monte Carlo numerical method for its calculation is presented. WRF-PartMC-LES is used to investigate impacts of emissions spatial heterogeneity in two parts: first, impacts on gas phase chemistry are explored with a particular focus on the O$_3$-NO$_x$-VOC system. We find that under high emissions spatial heterogeneity scenarios, O$_3$ concentrations in the PBL are up to 12\% lower than in a uniform emissions base case. Second, impacts of emissions spatial heterogeneity on the aerosol state are investigated, including effects on the aerosol number and mass distribution, composition, hygroscopicity, mixing state, and CCN activity. Under high emissions spatial heterogeneity scenarios, we find that CCN concentrations at low supersaturations (S=0.1--0.3\%) increase in the upper PBL by up to 25\% compared to a uniform emissions base case. This work is a first of a kind application of high resolution, particle-resolved LES for quantifying structural uncertainty of CCN activity due to the representation of emissions spatial heterogeneity.

\end{abstract}